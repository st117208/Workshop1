\usepackage{geometry}
\usepackage{amsmath,bm}
\usepackage{amssymb}
\usepackage{amsthm}
\usepackage{esint}
\usepackage{float}
\usepackage{graphicx}
\usepackage[dvipsnames]{xcolor}
\usepackage[utf8]{inputenc}
\usepackage[T2A]{fontenc}
%\usepackage[lutf8x]{luainputenc}

%\usepackage{fontspec}
%\setmainfont{CMU Serif}
%\setsansfont{CMU Sans Serif}
%\setmonofont{CMU Typewriter Text}

\usepackage[greek.polutoniko,english,russian]{babel}
% локализация и переносы
\usepackage{latexsym}
\usepackage{hhline}
\usepackage{multirow}
\usepackage{vmargin,indentfirst} 
\setmarginsrb{20mm}{15mm}{15mm}{15mm}{0mm}{0mm}{0mm}{10mm}
\usepackage[multiple]{footmisc}
\usepackage{hyperref}
\usepackage{multirow}
\usepackage{pdflscape}
%\numberwithin{equation}{section}
\usepackage[labelsep=period]{caption}


% команда для переноса строки внутри ячейки таблицы
\newcommand{\specialcell}[2][l]{%
  \begin{tabular}[#1]{@{}l@{}}#2\end{tabular}}

  \usepackage{physics}

\usepackage{tikz}
\usetikzlibrary{calc}
\usetikzlibrary{arrows.meta}

%\usepackage[margin=0.3in,bmargin=0.5in]{geometry}

%\usepackage{CJKutf8}%Chinese characters


\pagestyle{plain}

\theoremstyle{plain}
\newtheorem{theorem}{Theorem}[section]
\newtheorem{lemma}[theorem]{Lemma}
\newtheorem*{proposition}{Proposition}
\newtheorem{corollary}[theorem]{Corollary}
\newtheorem{problem}[theorem]{Problem}
\newtheorem{exercise}{Exercise}[section]

\theoremstyle{remark}
\newtheorem*{example}{Example}
\newtheorem*{examples}{Examples}
\newtheorem*{remark}{Remark}
\newtheorem*{hint}{Hint}

\theoremstyle{definition}
\newtheorem*{definition}{Definition}

\renewcommand{\emptyset}{\varnothing}
\newcommand\ph{\varphi}
\newcommand\eps{\varepsilon}

\newcommand\mbZ{\mathbb Z}
\newcommand\mbC{\mathbb C}
\newcommand\mbR{\mathbb R}
\newcommand\mbN{\mathbb N}
\newcommand\mbQ{\mathbb Q}
\newcommand\mbE{\mathbb E}

\newcommand\mcB{\mathcal B}
\newcommand\mcC{\mathcal C}
\newcommand\mcF{\mathcal F}
\newcommand\mcG{\mathcal G}
\newcommand\mcJ{\mathcal J}


\newcommand\la{\langle}
\newcommand\ra{\rangle}
\newcommand\ol{\overline}
\newcommand\ora{\overrightarrow}
\newcommand\rsa{\rightsquigarrow}

\DeclareMathOperator{\M}{M}
\DeclareMathOperator{\Span}{Span}
\DeclareMathOperator{\LL}{\cal L}
\DeclareMathOperator{\id}{id}
\DeclareMathOperator{\Ker}{Ker}
\DeclareMathOperator{\Img}{Im}
\DeclareMathOperator{\rk}{rank}
%\DeclareMathOperator{\Tr}{Tr}
\DeclareMathOperator{\cof}{cof}

\newcommand\dfn[1]{{\bf #1}}
\renewcommand{\thesubsubsection}{}

\newcounter{saveenumi}
\newcommand{\seti}{\setcounter{saveenumi}{\value{enumi}}}
\newcommand{\conti}{\setcounter{enumi}{\value{saveenumi}}}

\newcommand{\vct}[1]{\overrightarrow{#1}}
\newcommand{\bvct}[1]{\boldsymbol{#1}}
\newcommand{\codir}{\upuparrows}
\newcommand{\ncodir}{\uparrow \downarrow }

%% title background

\DeclareMathOperator{\sign}{sign}
\DeclareMathOperator{\const}{const}
\newcommand{\dprodd}[2]{\vct{#1}\cdot\vct{#2}}
\newcommand{\dprod}[2]{\bvct{#1}\cdot\bvct{#2}}

\newcommand{\cprod}[2]{\bvct{#1}\times\bvct{#2}}

\newcommand{\mprod}[3]{(\bvct{#1},\bvct{#2},\bvct{#3})}

\newcommand{\vecsys}{{$\bvct{a}_1$ , . . . , $\bvct{a}_n$ {}}}

\newcommand{\Problem}[1]{\vspace{\baselineskip}\textbf{Problem {#1}}\par}
\newcommand{\Solution}{\vspace{\baselineskip}\textbf{Solution}\par}

\newcommand*\coord[1]{%
  \begin{pmatrix}
    #1
  \end{pmatrix}%
}

\renewcommand{\d}[1]{\ensuremath{\operatorname{d}\!{#1}}}

\usepackage{tkz-euclide}
%\usetkzobj{all}

\usepackage{subcaption}

\usepackage{enumitem}

\usepackage{datetime}
\newdateformat{yeardate}{\THEYEAR}

\usepackage[toc,page]{appendix}

\def\contentsname{Содержание}

\usepackage{listings}

\lstset{% Собственно настройки вида листинга
inputencoding=utf8, extendedchars=\true, keepspaces = true, % поддержка кириллицы и пробелов в комментариях
language=C++,            % выбор языка для подсветки (здесь это Pascal)
basicstyle=\small\sffamily, % размер и начертание шрифта для подсветки кода
numbers=left,               % где поставить нумерацию строк (слева\справа)
numberstyle=\tiny,          % размер шрифта для номеров строк
stepnumber=1,               % размер шага между двумя номерами строк
numbersep=5pt,              % как далеко отстоят номера строк от подсвечиваемого кода
backgroundcolor=\color{white}, % цвет фона подсветки - используем \usepackage{color}
showspaces=false,           % показывать или нет пробелы специальными отступами
showstringspaces=false,     % показывать или нет пробелы в строках
showtabs=false,             % показывать или нет табуляцию в строках
frame=single,               % рисовать рамку вокруг кода
tabsize=2,                  % размер табуляции по умолчанию равен 2 пробелам
captionpos=t,               % позиция заголовка вверху [t] или внизу [b] 
breaklines=true,            % автоматически переносить строки (да\нет)
breakatwhitespace=false,    % переносить строки только если есть пробел
escapeinside={\%*}{*)}      % если нужно добавить комментарии в коде
}
